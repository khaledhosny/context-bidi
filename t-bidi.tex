%D \module
%D   [     file=t-bidi,
%D      version=0.003,
%D        title=BiDi,
%D     subtitle=Unicode BiDi for \CONTEXT,
%D       author=Khaled Hosny,
%D         date=\currentdate,
%D    copyright=Khaled Hosny,
%D      license=CC0]

%C To the extent possible under law, Khaled Hosny has waived all copyright and
%C related or neighboring rights to Bidi module. This work is published from:
%C Egypt.

%D \setupinteraction[state=start]
%D \setupcolors     [sttae=start]
%D \setupbodyfont   [sans]
%D
%D \useURL[UAX9][http://www.unicode.org/reports/tr9/]              [][Unicode Bidirectional Algorithm]
%D \useURL[CORR][http://www.unicode.org/versions/corrigendum6.html][][Unicode corrigendum \#6]
%D
%D This modules provides an implementation of \from[UAX9] as well as a higher
%D level interface of \LUATEX's bidirectional support.
%D
%D \subsubject{Options}
%D
%D \loadsetups[t-bidi.xml]
%D \showsetup{setupbidi}
%D
%D There are currently no much options, one can only set the main and math
%D directions.
%D
%D \subject{Implementation}

\writestatus{loading}{Unicode Bidirectional Algorithm}

\startmodule[bidi]

\unprotect

%D Auxiliary functions used by the \LUA\ module to access \CONTEXT's Unicode
%D character database.

\startluacode
bidi            = bidi or { }
local chardata  = characters.data

function bidi.get_direction(c)
    local dir = chardata[c] and chardata[c].direction or "l"
    return dir
end

function bidi.get_mirror(c)
    local mir = chardata[c] and chardata[c].mirror
    return mir
end

local function set_mirror(c, m)
    if chardata[c] then
        chardata[c].mirror = m
    end
end
\stopluacode

%D Typographic quotes do not have mirrored property in Unicode for stupid
%D backward reasons (see \from[CORR]).

\startluacode
for _,v in next, { 0x2018, 0x201C, 0x301D } do
    set_mirror(v, v+1)
end
for _,v in next, { 0x2019, 0x201D, 0x301E } do
    set_mirror(v, v-1)
end
\stopluacode

%D Load the \LUA\ module

\registerctxluafile{bidi}{0.002}

%D Setups

\startluacode
function bidi.setup(str)
    local settings = { }

    utilities.parsers.settings_to_hash(str, settings)

    for k,v in next, settings do
        if k == "main" then
            bidi.maindir = v
            if v == "r2l" then
                context.setupalign{"r2l"}
            end
	elseif k == "math" and v == "r2l" then
            context.setupmathematics{align="r2l"}
        end
    end
end

nodes.tasks.appendaction("processors",   "characters",  "bidi.process")
nodes.tasks.appendaction("vboxbuilders", "normalizers", "bidi.process_align")
nodes.tasks.appendaction("math",         "after",       "bidi.process_math")
\stopluacode

\unexpanded\def\setupbidi
  {\dosingleargument\dosetupbidi}

\def\dosetupbidi[#1]%
  {\ctxlua{bidi.setup("#1")}}

%D Defaults

\setupbidi
[
  main=r2l,
  math=l2r,
]

\protect

\stopmodule

\continueifinputfile{t-bidi.tex}

\usemodule[bidi]
\usemodule[simplefonts]
\setmainfont[dejavusans][features=arabic]
\setupbidi[main=r2l, math=r2l]

\starttext

\def\TEST{أبجد () هوز ab()cd حطي g كلمن ١٢٣٤٥ سعفص قرشت.}

\TEST

أبجد () هوز ab()cd حطي g كلمن {\textdir TRT ١٢٣٤٥} سعفص {\textdir TLT قرشت}.

$123 + 456 = (579) \sum^{123}_{1+2+3}$
$$123 + 456 = (579) \sum^{123}_{1+2+3}$$

\placetable[here,force,nonumber]{}
\starttable[|rp(.45\textwidth)|rp(.45\textwidth)|]
\NC أول \TEST \NC ثان \TEST \FR
\stoptable

\stoptext

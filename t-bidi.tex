%D \module
%D   [     file=t-bidi,
%D      version=0.003,
%D        title=Bidi Processing,
%D     subtitle=Higher level interface to \LUATEX's bidirectional support,
%D       author=Khaled Hosny,
%D         date=\currentdate,
%D    copyright=Khaled Hosny,
%D      license=CC0]

\writestatus{loading}{Unicode Bidirectional Algorithm}

\startmodule[bidi]

\unprotect

\startluacode

bidi            = bidi or { }
local chardata  = characters.data

function bidi.get_direction(c)
    local dir = chardata[c] and chardata[c].direction or "l"
    return dir
end

function bidi.get_mirror(c)
    local mir = chardata[c] and chardata[c].mirror
    return mir
end

local function set_mirror(c, m)
    if chardata[c] then
        chardata[c].mirror = m
    end
end

-- see http://www.unicode.org/versions/corrigendum6.html
for _,v in next, { 0x2018, 0x201C, 0x301D } do
    set_mirror(v, v+1)
end
for _,v in next, { 0x2019, 0x201D, 0x301E } do
    set_mirror(v, v-1)
end

\stopluacode

\registerctxluafile{bidi}{0.002}

\startluacode

function bidi.setup(str)
    local settings = { }

    utilities.parsers.settings_to_hash(str, settings)

    for k,v in next, settings do
        if k == "main" then
            bidi.maindir = v
        end
    end
end

nodes.tasks.appendaction("processors",   "characters",  "bidi.process")
nodes.tasks.appendaction("vboxbuilders", "normalizers", "bidi.process_align")
nodes.tasks.appendaction("math",         "after",       "bidi.process_math")

\stopluacode

\unexpanded\def\setupbidi
  {\dosingleargument\dosetupbidi}

\def\dosetupbidi[#1]%
  {\ctxlua{bidi.setup("#1")}}

\protect

\stopmodule

%\continueifinputfile{t-bidi.tex} % does not work
\doifnotmode{demo}{\endinput}

\usemodule[bidi]
\usemodule[simplefonts]
\setmainfont[dejavusans][features=arabic]
\setupbidi[main=r2l]
\setupalign[r2l]
\setupmathematics[align=r2l]

\starttext

\def\TEST{أبجد () هوز ab()cd حطي g كلمن ١٢٣٤٥ سعفص قرشت.}

\TEST

أبجد () هوز ab()cd حطي g كلمن {\textdir TRT ١٢٣٤٥} سعفص {\textdir TLT قرشت}.

$123 + 456 = (579) \sum^{123}_{1+2+3}$
$$123 + 456 = (579) \sum^{123}_{1+2+3}$$

\placetable[here,force,nonumber]{}
\starttable[|rp(.45\textwidth)|rp(.45\textwidth)|]
\NC أول \TEST \NC ثان \TEST \FR
\stoptable

\stoptext
